\documentclass[11pt]{article}
\usepackage{graphicx,bm,amssymb,amsmath,amsthm}
% timo macros...
\newcommand{\bx}{\textbf{x}}
\newcommand{\by}{\textbf{y}}
\newcommand{\bs}{\textbf{s}}

% alex macros...
\newcommand{\bi}{\begin{itemize}}
\newcommand{\ei}{\end{itemize}}
\newcommand{\ben}{\begin{enumerate}}
\newcommand{\een}{\end{enumerate}}
\newcommand{\be}{\begin{equation}}
\newcommand{\ee}{\end{equation}}
\newcommand{\bea}{\begin{eqnarray}}
\newcommand{\eea}{\end{eqnarray}}
\newcommand{\bc}{\begin{center}}
\newcommand{\ec}{\end{center}}
\newcommand{\bfi}{\begin{figure}}
\newcommand{\efi}{\end{figure}}
\newcommand{\ca}[2]{\caption{#1 \label{#2}}}
\newcommand{\ig}[2]{\includegraphics[#1]{#2}}
\newcommand{\spl}[2]{\left\{\begin{array}{ll}#1\\#2\end{array}\right.}
\newcommand{\ie}{{\it i.e.\ }}
\newcommand{\eg}{{\it e.g.\ }}
\newcommand{\pd}[2]{\frac{\partial #1}{\partial #2}}
\newcommand{\pdc}[3]{\left. \frac{\partial #1}{\partial #2}\right|_{#3}}
\newcommand{\infint}{\int_{-\infty}^{\infty} \!\!}      % infinite integral
\newcommand{\tbox}[1]{{\mbox{\tiny #1}}}
\newcommand{\mbf}[1]{{\bm #1}}           % requires bm package
\newcommand{\pO}{{\partial\Omega}}
\newcommand{\uN}{u^{(N)}}
\newcommand{\emach}{\epsilon_\tbox{mach}}
%\newcommand{\crk}{C_{s}} %C_{R,k}         % const for exponential s(m) bnd
\newcommand{\bmp}[1]{\begin{minipage}{#1}}
\newcommand{\emp}{\end{minipage}}
\newcommand{\re}{\mbox{Re}\,}
\newcommand{\im}{\mbox{Im}\,}

\newtheorem{thm}{Theorem}
\newtheorem{cnj}{Conjecture}
\newtheorem{lem}{Lemma}[thm]
\newtheorem{cor}{Corollary}[thm]

\newtheorem{rmk}[thm]{Remark}
\newtheorem{pro}[thm]{Proposition}
%\newtheorem{cnj}[thm]{Conjecture}
%\newtheorem{lemma}[thm]{Lemma}

\newcommand{\bp}{{\bf Proof:\ }}   % {\begin{proof}}
\newcommand{\ep}{\hfill $\square$ \vspace{1ex} \\} % {\end{proof}}
% note elsart ones horrible: \pr and \endpr
\newcommand{\bmal}{{\bm \alpha}}             % bold alpha
\newcommand{\splt}[2]{$\left\{\begin{array}{l}\mbox{#1}\\\mbox{#2}\end{array}\ri
ght.$}
\newcommand{\lo}{{L^2(\Omega)}}
\newcommand{\lpo}{{L^2(\pO)}}
\newcommand{\cu}{\tilde{u}}            % symbol for continuation of u
\newcommand{\cOmega}{\tilde{\Omega}}   % symbol for domain to be continued into
\newcommand{\dmax}{D_\tbox{max}}       % max dist in splane_charge_curve

% spacers...
\newcommand{\tw}{\textwidth}
\newcommand{\vg}{\vspace{2ex}}

% brand names... the problem with putting a trailing space here is that then
% punctuation following the word cannot be correct. To use these with correct
% trailing space, use as \mpspack\ or \matlab\ or with ending period, \matlab.
\newcommand{\mpspack}{{\tt MPSpack}}            % name of this package
\newcommand{\matlab}{{M{\small ATLAB}}}            % name of MATLAB

% code formatting... (nearly ok, gets multiple-line padding wrong)
\newcommand{\co}[1]{\vspace{.5ex}
\\
\mbox{}\hspace{5ex}\begin{minipage}{\textwidth}{\tt #1}\end{minipage}%
\vspace{.5ex}%
\\}



\begin{document}
\title{\mpspack\ tutorial}
\author{Alex Barnett\footnote{Department of Mathematics, Dartmouth College,
Hanover, NH, 03755, USA}
\ and
Timo Betcke\footnote{Department of Mathematics,
University of Reading, Berkshire, RG6 6AX, UK}}
\date{\today}   % how pipe from getrevisionnumber + 1?

\maketitle
\begin{abstract}
This is a short tutorial showing how boundary-value problems
may be simply and accurately solved with the \mpspack\ toolbox
in \matlab. We assume basic familiarity with \matlab\
and with partial differential equations.
%concept of particular-solution type numerical methods is assumed.
\end{abstract}

\tableofcontents

\section{About this tutorial}

This tutorial is designed for `bottom-up' learning of the features
of \mpspack, i.e.\ by progressing through simple examples.
In that sense it complements the user manual which describes
the theoretical framework in broad strokes and therefore could
be considered `top-down'.
We will skip the mathematics behind the solution techniques,
focusing on computing and plotting useful PDE solutions.

Throughout we will identify the plane $\mathbb{R}^2$ with the complex
plane $\mathbb{C}$, by the usual map $z=x+iy$. In other words
$(2,3)$ and $2+3i$ represent the same point.
We use {\tt teletype} font to designate commands that may be typed
at the \matlab\ prompt.
All the code examples in this document, and code to generate the
figures, is found in {\tt tutorial.m} in the {\tt examples/} directory.

\bfi % ffffffffffffffffffffffffffffffffffffffffffffffffffffffffffffffffffffff
a)\raisebox{-0.35\textwidth}{\ig{height=0.35\textwidth}{seg.eps}}
b)\raisebox{-0.35\textwidth}{\ig{height=0.35\textwidth}{dom.eps}}
\ca{a) circular closed segment, b) unit disc domain.
Both have a periodic trapezoidal quadrature rule
with $M=20$ quadrature points}{f:sd}
\efi

\section{Solving Laplace's equation in a smooth domain}

We start by setting up a domain in $\mathbb{R}^2$.
% in which a PDE will be solved.
Domains are built from segments which define their boundary.
To make the unit disc domain,
we first need a circle segment with center
0, radius 1, and angle range
$[0,2\pi)$, as follows,
\co{s = segment([], [0 1 0 2*pi])}
The object {\tt s} is indeed a circular segment, as we may check by
typing {\tt s.plot}, producing Fig.~\ref{f:sd}a.
%or by examining its contents by typing {\tt s}.
All segments have a {\em sense}, i.e.\ direction of travel:
for this segment it is counter-clockwise, as shown by the
downwards-pointing
arrow symbol overlayed onto the segment at about 9 o'clock.%
  \footnote{In fact, segments are parametrized internally as function $z(t)$
    of a real variable $t\in[0,1]$, and the sense is the direction of
    increasing $t$. Segment {\tt s} stores this function as {\tt s.Z}.}
Notice also normal vectors (short `hairs') pointing outwards
at each boundary point; our definition is that
normals on a segment always point to the {\em right} when traversing the
sense of the segment.

We create the domain interior to this segment with
\co{d = domain(s, +1)}
where the second argument (here $+1$, the only other option being $-1$)
specifies that the domain is to the `standard' side of the segment, which
we take to be such that the normals point {\em away from} the domain.
That is, with $+1$ the domain lies to the {\em left} of the segment
when traversed in its correct sense (with $-1$ the domain
would lie to the right of the segment.)
Typing {\tt d.plot} produces%
  \footnote{There are extra plotting options and features that
    are described in documentation such as {\tt help domain.plot}.
    E.g.\ in this figure a grid of points interior to the domain has been
    included, achieved with {\tt opts.gridinside=0.05; d.plot(opts);}
    %TODO: demo switch off corner fan.
  }
Fig.~\ref{f:sd}b.
Note that perimeter and area are automatically
labelled (these are only rough approximations intended for sanity checks).

%The small black semicircle at angle $\theta=0$ illustrates that \mpspack\
%has connected the start and end of the segment to form a `corner'
%(which in this case has an 

Laplace's equation $\Delta u = 0$ is Helmholtz's equation with wavenumber
zero, which we set for this domain with,
\co{d.k = 0;}
Our philosophy is
to approximate the solution in the domain by a linear combination of
{\em basis functions}, each defined over the whole domain.
%We have to choose how many to use, i.e.\ the
%{\em order} of approximation.
We choose harmonic polynomials up to 8th order,
i.e. $u(z) = \sum_{n=0}^{8} c_n \,\mbox{Re}\,z^n +
\sum_{n=1}^{8} c_{-n}\,\mbox{Im}\,z^n$, where $\mbf{c}:=\{c_n\}_{n=-8}^{8}
\in\mathbb{R}^{17}$
is a coefficient vector,
%i.e.\ the sets $\{\mbox{Re } z^n\}_{n=0}^{10}$ and
%$\{\mbox{Im } z^n\}_{n=1}^{10}$,
using the command
\co{d.addregfbbasis([], 8);}
Let's specify Dirichlet boundary data $f(x,y) = \ln \sqrt{(x-2)^2+(y-3)^2} =
\ln |z-2-3i|$, for
$z$ on the segment, by representing this as an anonymous function {\tt f}
and passing it to the segment,
\co{f = @(z) log(abs(z-2-3i));  \\
s.setbc(-1, 'd', [], @(t) f(s.Z(t)));}
Note that in fact we needed to pass in a function of the segment parameter $t$,
which was achieved by
wrapping ${\tt f}$ around the parametrization function ${\tt s.Z}$.
The first argument $-1$ expresses that the boundary condition is to be
understood in the limit approaching from the side {\em opposite} the
segment's normal direction, which is where the domain is located.
Finally we set up a BVP by passing domains to a problem object, and then
may solve (in the least-squares sense) for the coefficients
\co{p = bvp(d);\\
p.solvecoeffs;}
Now {\tt p.co} contains the coefficients vector $\mbf{c}$.
To evaluate and the solution we simply use,
\co{p.showsolution;}
This command evaluated selected an appropriate grid covering the domain
(points outside the domain are made transparent), giving Fig.~\ref{f:u}a.

\bfi % ffffffffffffffffffffffffffffffffffffffffffffffffffffffffffffffffffffff
a)\raisebox{-0.35\textwidth}{\ig{height=0.35\textwidth}{u.eps}}
b)\raisebox{-0.35\textwidth}{\ig{height=0.35\textwidth}{uerr.eps}}
\ca{a) numerical solution field $u$, b) pointwise error $u-f$,
for Laplace's equation in the unit disc with $M=20$ quadrature points
and harmonic polynomials up to 8th order.}{f:u}
\efi

How accurate was our numerical solution $u$? One measure is the
$L^2$ error on the boundary, and is estimated by
\co{p.bcresidualnorm}
However, since the function $f(z)$ is already harmonic in $\mathbb{R}^2$,
it is the unique analytic solution, and we may compare against this
over the domain by passing in an option,
\co{opts.comparefunc = f; p.showsolution(opts);}
giving Fig.~\ref{f:u}b.
%We may collect the gridded solution $u$ and subtract it from $f$ evaluated
%on the same grid to plot the error 
%\co{[u gx gy di] = p.showsolution;}

\subsection{Convergence}

This is easy.
Change $M$, watch 
\co{pr.bcresidualnorm}
Convergence loop with $N$, easy.


\section{More interesting domains, and corners}

We create a closed segment with polar function $r(\theta) = 1 + \cos 3\theta$
for $0\le \theta<2\pi$ using
\co{s = segment.radialfunc([], {@(th) 1 + cos(3*th), @(th) -3*sin(3*th)});}
Notice that we pass in a cell array of two function handles:
$r(\theta)$ and $r'(\theta)$. The latter is needed to compute normal
directions accurately.




\section{Exterior domains}

\section{Scattering problems}

\section{Corner basis sets}




\bibliographystyle{siam} 
\bibliography{alexrefs}
\end{document}
