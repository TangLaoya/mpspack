\documentclass[12pt]{article}
\usepackage{graphicx,bm,amssymb,amsmath,amsthm}
% timo macros...
\newcommand{\bx}{\textbf{x}}
\newcommand{\by}{\textbf{y}}
\newcommand{\bs}{\textbf{s}}

% alex macros...
\newcommand{\bi}{\begin{itemize}}
\newcommand{\ei}{\end{itemize}}
\newcommand{\ben}{\begin{enumerate}}
\newcommand{\een}{\end{enumerate}}
\newcommand{\be}{\begin{equation}}
\newcommand{\ee}{\end{equation}}
\newcommand{\bea}{\begin{eqnarray}}
\newcommand{\eea}{\end{eqnarray}}
\newcommand{\bc}{\begin{center}}
\newcommand{\ec}{\end{center}}
\newcommand{\bfi}{\begin{figure}}
\newcommand{\efi}{\end{figure}}
\newcommand{\ca}[2]{\caption{#1 \label{#2}}}
\newcommand{\ig}[2]{\includegraphics[#1]{#2}}
\newcommand{\spl}[2]{\left\{\begin{array}{ll}#1\\#2\end{array}\right.}
\newcommand{\ie}{{\it i.e.\ }}
\newcommand{\eg}{{\it e.g.\ }}
\newcommand{\pd}[2]{\frac{\partial #1}{\partial #2}}
\newcommand{\pdc}[3]{\left. \frac{\partial #1}{\partial #2}\right|_{#3}}
\newcommand{\infint}{\int_{-\infty}^{\infty} \!\!}      % infinite integral
\newcommand{\tbox}[1]{{\mbox{\tiny #1}}}
\newcommand{\mbf}[1]{{\bm #1}}           % requires bm package
\newcommand{\pO}{{\partial\Omega}}
\newcommand{\uN}{u^{(N)}}
\newcommand{\emach}{\epsilon_\tbox{mach}}
%\newcommand{\crk}{C_{s}} %C_{R,k}         % const for exponential s(m) bnd
\newcommand{\bmp}[1]{\begin{minipage}{#1}}
\newcommand{\emp}{\end{minipage}}
\newcommand{\re}{\mbox{Re}\,}
\newcommand{\im}{\mbox{Im}\,}

\newtheorem{thm}{Theorem}
\newtheorem{cnj}{Conjecture}
\newtheorem{lem}{Lemma}[thm]
\newtheorem{cor}{Corollary}[thm]

\newtheorem{rmk}[thm]{Remark}
\newtheorem{pro}[thm]{Proposition}
%\newtheorem{cnj}[thm]{Conjecture}
%\newtheorem{lemma}[thm]{Lemma}

\newcommand{\bp}{{\bf Proof:\ }}   % {\begin{proof}}
\newcommand{\ep}{\hfill $\square$ \vspace{1ex} \\} % {\end{proof}}
% note elsart ones horrible: \pr and \endpr
\newcommand{\bmal}{{\bm \alpha}}             % bold alpha
\newcommand{\splt}[2]{$\left\{\begin{array}{l}\mbox{#1}\\\mbox{#2}\end{array}\ri
ght.$}
\newcommand{\lo}{{L^2(\Omega)}}
\newcommand{\lpo}{{L^2(\pO)}}
\newcommand{\cu}{\tilde{u}}            % symbol for continuation of u
\newcommand{\cOmega}{\tilde{\Omega}}   % symbol for domain to be continued into
\newcommand{\dmax}{D_\tbox{max}}       % max dist in splane_charge_curve

% spacers...
\newcommand{\tw}{\textwidth}
\newcommand{\vg}{\vspace{2ex}}

% brand names... the problem with putting a trailing space here is that then
% punctuation following the word cannot be correct. To use these with correct
% trailing space, use as \mpspack\ or \matlab\ or with ending period, \matlab.
\newcommand{\mpspack}{{\tt MPSpack}}            % name of this package
\newcommand{\matlab}{{M{\small ATLAB}}}            % name of MATLAB

% code formatting... (nearly ok, gets multiple-line padding wrong)
\newcommand{\co}[1]{\vspace{.5ex}
\\
\mbox{}\hspace{5ex}\begin{minipage}{\textwidth}{\tt #1}\end{minipage}%
\vspace{.5ex}%
\\}



\begin{document}
\title{\mpspack\ user manual}
\author{Alex Barnett\footnote{Department of Mathematics, Dartmouth College, Hanover, NH, 03755, USA}
and Timo Betcke\footnote{Department of Mathematics,
University of Reading, Berkshire, RG6 6AX, UK}}
\date{\today}   % how pipe from getrevisionnumber + 1?

\maketitle
\begin{abstract}
\mpspack\ is a fully object-oriented \matlab\ toolbox for
solving Laplace, Helmholtz, wave scattering, and related PDE boundary-value
problems
on piecewise-homogeneous 2D domains.
The philosophy is to
%represent solutions using
use basis functions
which are particular solutions to the
PDE in some region; solving is thus reduced to matching on
the boundary (or on boundaries of subregions).
This idea is known as the Method of Particular Solutions, or as Trefftz,
ultra-weak, or non-polynomial 
methods in the FEM community.
Basis functions include plane-wave, Fourier-Bessel,
corner-adapted expansions,
fundamental solutions.
Layer potential representations and associated
singular quadrature schemes are also integrated into this framework.
It is designed to
be simple to use, and to enable highly-accurate solutions.  This is
the user manual; for a more hands-on approach try the accompanying tutorial.
\end{abstract}

\section{Overview}

In numerical analysis there has been a recent flurry of activity
on methods for solving PDEs where solutions are approximated by
linear combinations of particular solutions to the PDE with
good convergence properties.
These methods are high-order (often exponentially convergent),
efficient at high frequencies (the number of degrees of freedom
scales linearly with wavenumber, in 2D),
and are quite simple to implement.
When geometries become more complicated, the domain needs to be split
up into multiple subdomains, e.g.\ one for each corner, and the
implementation becomes cumbersome.
The goal of this software toolbox is to make such an implementation simple
and transparent, and create an intuitive, unifying framework in which many
types of BVP, basis sets, boundary conditions, and domain geometries
may be solved, explored and graphed with ease.
Since in these methods, the number of subdomains is small and fixed,
we leave the job of specifying subdomains, i.e.\ `meshing',
to the user (this contrasts with the philosophy of the large number of
finite-element packages already in existence.)

We are most interested in the scalar Helmholtz equation in the plane,
\bea
(\Delta+k^2)u& =& 0 \qquad \mbox{in } \Omega~,
\eea
where $\Omega \subset \mathbb{R}^2$ is an interior or exterior
domain, $k$ is the wavenumber, and certain
imhomogeneous boundary conditions are imposed.
Such problems arise in wave scattering and cavity resonances.
More generally we may have multiple domains with different
wavenumbers connected by homogeneous or inhomogeneous boundary conditions,
as in transmission, dielectric-coated, or photonic crystal problems.
With $k=0$ we have Laplace's equation, with applications to
electrostatics, steady-state heat flow, and probability.

In this release we discuss BVPs.
It is a small extension to compute
eigenvalue problems and periodic problems; we will document these in
future releases.
%We believe the object-oriented framework we have set up generalizes
%gracefully to 

Fig.~\ref{f:relat} shows how a geometry is built up from objects, and the
relationships between them.

Our influences include Driscoll's SC Toolbox, and the {\tt chebfun} system
by Trefethen et al.


\bfi % fffffffffffffffffffffffffff
\ig{width=\textwidth}{relat.eps}
\ca{Relationship of {\tt MPSpack}
objects used to solve a typical BVP or EVP.
The physical geometry is shown on the left, and the code structures on the
right.
Each segment has a normal direction. Each domain is built from
segments with a list of signs ($+$ or $-$) which define the sense in which
the segments are used to form the boundary.
Basis sets affect the function values inside each domain in which they
are referenced. A basis set may influence more than one domain.}{f:relat}
\efi

\section{Installation}

Requirements:
\bi
\item \matlab\ version 7.6 (2008a) or newer is needed,
since we make heavy use of recent object-oriented programming features.
\ei

The project is hosted at
\co{http://code.google.com/p/mpspack}
There are two alternative methods to download and unpack:
\ben
\item
Get a gzip-compressed tar archive from
\co{http://code.google.com/p/mpspack/downloads/list}
In a UNIX environment you may now
unpack this with
\co{tar zxvf mpspack-r*.tar.gz}
This creates the directory {\tt mpspack-r*} containing the toolbox,
where {\tt *} represents the revision (version) number.

\item
First install the {\tt subversion} (SVN) version
control software from
\co{http://subversion.tigris.org}
Anonymous check out of {\tt MPSpack} is then via the subversion command:
\co{svn checkout http://mpspack.googlecode.com/svn/trunk/ mpspack}
This creates a directory {\tt mpspack} containing the toolbox.

You might prefer a more user-friendly graphical
subversion client such as those listed at
\co{http://subversion.tigris.org/links.html\#clients}
\een

There are some optional fast basis evaluation routines
(C and Fortran with MEX interfaces), which should be compiled
in a UNIX environment as follows:
from the {\tt mpspack} directory type {\tt make}

You should now add the {\tt mpspack}
subdirectory to your MATLAB path, for instance by adding the line
\co{addpath 'path/to/mpspack';}
to your MATLAB {\tt startup.m} file.
You are now ready to use \mpspack\ !


\subsection{Boundary-value problems}

Let $\Omega \subset \mathbb{R}^2$

\bea
(\Delta+k^2)u& =& 0 \qquad \mbox{in } \Omega\\
au + b u_n&=&f\qquad \mbox{on } \pO
\eea

In our implementation we restrict to functions $a$, $b$ that are
constant on segments used to define the boundary $\pO$.

If $\Omega$ is an exterior domain, we may wish to impose
additional boundary conditions at infinity, such as Sommerfeld's
radiation condition.
This may be achieved by choosing basis sets satisfying this condition.

See \cite{mfs} for
an example of using the Method of Fundamental Solutions to solve
an interior Helmholtz problem.

To solve a BVP the usual flow is as follows:

\ben
\item define piecewise-analytic segments forming all boundaries
\item define domains using various of these segments as their boundaries
\item choose MPS basis set(s) within each domain
\item set up inhomogeneous boundary or matching conditions on each segment
\item solve a %least-squares
linear system to get the basis coefficient vector
\item evaluate solution on desired points
\item plot solution or compute error estimates
\een

\ldots

\subsection{Eigenvalue problems}




\subsection{Geometry}

\ldots

When domains are built from segments, each segment stores information about
which domain, if any, it is connected to on each side.
If segments are to be reused to construct new domains, while erasing any
old domains, the current way to do this is via the {\tt s.disconnect}
command, where {\tt s} is a segment handle or array of such. This clears
these segments of any connections with domains.

The segment object field responsible for
this is {\tt s.dom}, which is a 1-by-2 cell array
of handles to the domain on the segment's natural normal side and
back side respectively. If there is no domain on a particular side, the
cell element is empty.


\subsection{Boundary conditions}

A segment $\Gamma$ may contain a boundary condition on only one of its sides,
of the form
$$
a u(s) + b u_n(s) = f(s)  \qquad s\in\Gamma, \quad a,b \in \mathbb{C}
$$
or, alternatively matching conditions of the form
\bea
%\left.\begin{array}{rcl}
a^+ u^+(s) + a^- u^-(s) &=& f(s)\\
b^+ u_n^+(s) + b^- u_n^-(s) &=& g(s)
\eea
where $a^+,a^-,b^+,b^- \in \mathbb{C}$, and where $u^+$ is the
limiting value approaching the segment from its positive normal side.
In the above $f$ and $g$ are functions on the boundary.
In our implementation the user
may supply these as handles to functions of segment parameter
$t\in[0,1]$, or as a data column vector on the segment quadrature points.

The segment fields {\tt a} and {\tt b} store
$a$ and $b$ for a boundary condition, or
$[a^+, a^-]$ and $[b^+, b^-]$
respectively for a matching condition.

\subsection{Basis sets}

Basis sets exist independently of domains or segments
\footnote{apart from layer potentials which require a segment on which
their source density is placed}.
However, for a basis set to influence a domain {\tt d}, its handle must appear
in the cell array field {\tt d.bas}
This is set up automatically when you perform {\tt d.addregfbbasis()}
or any of the other 

There will be more elaborate basis set types which naturally affect two domains,
such as dielectric corner and transmission layer potential objects;
their handles will appear in the {\tt bas} cell array of more than one domain.


\section{In-depth examples}

\section{Test routines}

The code {\tt testbvp.m} shows the main steps for solution of a BVP on
a variety of domains.

\section{Limitations}

\bi
\item Checking of whether inside a domain is approximate, based on approximating
polygons.
\ei

\section{List of improvements to make}

List to be prioritized:

\bi
\item make {\tt segment.bdryfunc} which creates $u$, $u_n$ data given a
field function and its gradient field function, useful for choosing BC
data corresponding to exact fields
\item make {\tt domain.setbc} which uses one BC data function on all segments
\item eigenvalue problems: MPS, scaling method
\item segment methods to create analytic interpolant function from boundary
point data, enabling user to specify a segment using points on a curve.
\item add better automated ways to choose MFS charge points, based on JCP paper
\item add layer-potential bases, and associated periodic quadrature methods
\item write {\tt segment.bdrysolution} which evaluates $u$, $u_n$ on one
or other side of a boundary. For layer potentials this would take into account
jump relations. This should then be used in {\tt problem.fillbcmatrix}
\item keep discrepancy and evaluation matrices, for efficiency in multiple
right-hand sides.
\item classes which symmetrize basis sets for single reflection, $C_4$, etc
symmetry, by wrapping the calls to basis evaluations using reflection
points
\item dielectric constants, overall frequency,
automatically changing $k$ in domains
\item periodic boundary conditions connecting values and derivatives
on given sides of two different segments, needed for photonic crystals
\item allow nufbbasis to accept empty k input, like other basis types.
\item make better addnufbbasis which attaches to a requested corner.
\item basisgeom for nufbbasis, split off utils.plotfan from domain.plot and
use.
\ei

\section{Known bugs}

Please alert the authors to any bugs you discover, including a description
of how to reproduce the behavior, using the interface at

{\tt http://code.google.com/p/mpspack/issues/list}

\bibliographystyle{siam} 
\bibliography{alexrefs}



\end{document}
