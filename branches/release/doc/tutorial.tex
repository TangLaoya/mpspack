\documentclass[11pt]{article}
\usepackage{graphicx,bm,amssymb,amsmath,amsthm}
% timo macros...
\newcommand{\bx}{\textbf{x}}
\newcommand{\by}{\textbf{y}}
\newcommand{\bs}{\textbf{s}}

% alex macros...
\newcommand{\bi}{\begin{itemize}}
\newcommand{\ei}{\end{itemize}}
\newcommand{\ben}{\begin{enumerate}}
\newcommand{\een}{\end{enumerate}}
\newcommand{\be}{\begin{equation}}
\newcommand{\ee}{\end{equation}}
\newcommand{\bea}{\begin{eqnarray}}
\newcommand{\eea}{\end{eqnarray}}
\newcommand{\bc}{\begin{center}}
\newcommand{\ec}{\end{center}}
\newcommand{\bfi}{\begin{figure}}
\newcommand{\efi}{\end{figure}}
\newcommand{\ca}[2]{\caption{#1 \label{#2}}}
\newcommand{\ig}[2]{\includegraphics[#1]{#2}}
\newcommand{\spl}[2]{\left\{\begin{array}{ll}#1\\#2\end{array}\right.}
\newcommand{\ie}{{\it i.e.\ }}
\newcommand{\eg}{{\it e.g.\ }}
\newcommand{\pd}[2]{\frac{\partial #1}{\partial #2}}
\newcommand{\pdc}[3]{\left. \frac{\partial #1}{\partial #2}\right|_{#3}}
\newcommand{\infint}{\int_{-\infty}^{\infty} \!\!}      % infinite integral
\newcommand{\tbox}[1]{{\mbox{\tiny #1}}}
\newcommand{\mbf}[1]{{\bm #1}}           % requires bm package
\newcommand{\pO}{{\partial\Omega}}
\newcommand{\uN}{u^{(N)}}
\newcommand{\emach}{\epsilon_\tbox{mach}}
%\newcommand{\crk}{C_{s}} %C_{R,k}         % const for exponential s(m) bnd
\newcommand{\bmp}[1]{\begin{minipage}{#1}}
\newcommand{\emp}{\end{minipage}}
\newcommand{\re}{\mbox{Re}\,}
\newcommand{\im}{\mbox{Im}\,}

\newtheorem{thm}{Theorem}
\newtheorem{cnj}{Conjecture}
\newtheorem{lem}{Lemma}[thm]
\newtheorem{cor}{Corollary}[thm]

\newtheorem{rmk}[thm]{Remark}
\newtheorem{pro}[thm]{Proposition}
%\newtheorem{cnj}[thm]{Conjecture}
%\newtheorem{lemma}[thm]{Lemma}

\newcommand{\bp}{{\bf Proof:\ }}   % {\begin{proof}}
\newcommand{\ep}{\hfill $\square$ \vspace{1ex} \\} % {\end{proof}}
% note elsart ones horrible: \pr and \endpr
\newcommand{\bmal}{{\bm \alpha}}             % bold alpha
\newcommand{\splt}[2]{$\left\{\begin{array}{l}\mbox{#1}\\\mbox{#2}\end{array}\ri
ght.$}
\newcommand{\lo}{{L^2(\Omega)}}
\newcommand{\lpo}{{L^2(\pO)}}
\newcommand{\cu}{\tilde{u}}            % symbol for continuation of u
\newcommand{\cOmega}{\tilde{\Omega}}   % symbol for domain to be continued into
\newcommand{\dmax}{D_\tbox{max}}       % max dist in splane_charge_curve

% spacers...
\newcommand{\tw}{\textwidth}
\newcommand{\vg}{\vspace{2ex}}

% brand names... the problem with putting a trailing space here is that then
% punctuation following the word cannot be correct. To use these with correct
% trailing space, use as \mpspack\ or \matlab\ or with ending period, \matlab.
\newcommand{\mpspack}{{\tt MPSpack}}            % name of this package
\newcommand{\matlab}{{M{\small ATLAB}}}            % name of MATLAB

% code formatting... (nearly ok, gets multiple-line padding wrong)
\newcommand{\co}[1]{\vspace{.5ex}
\\
\mbox{}\hspace{5ex}\begin{minipage}{\textwidth}{\tt #1}\end{minipage}%
\vspace{.5ex}%
\\}



\begin{document}
\title{\mpspack\ tutorial}
\author{Alex Barnett\footnote{Department of Mathematics, Dartmouth College,
Hanover, NH, 03755, USA}
\ and
Timo Betcke\footnote{Department of Mathematics,
University of Reading, Berkshire, RG6 6AX, UK}}
\date{\today}   % how pipe from getrevisionnumber + 1?

\maketitle
\begin{abstract}
This is a short tutorial showing how boundary-value problems
may be numerically solved simply and accurately with the \mpspack\ toolbox
for \matlab. We assume basic familiarity with \matlab\
and with partial differential equations.
%concept of particular-solution type numerical methods is assumed.
\end{abstract}

%\tableofcontents

\section{About this tutorial}

This tutorial is designed for `bottom-up' learning of the features
of \mpspack, i.e.\ by progressing through simple examples.
In that sense it complements the user manual which describes
the theoretical framework in broad strokes and therefore could
be considered `top-down'.
We will skip the mathematics behind the solution techniques,
focusing on computing and plotting useful PDE solutions.

Throughout we will identify the plane $\mathbb{R}^2$ with the complex
plane $\mathbb{C}$, by the usual map $z=x+iy$. In other words
$(2,3)$ and $2+3i$ represent the same point.
We use {\tt teletype} font to designate commands that may be typed
at the \matlab\ prompt.
All the code examples in this document, and code to generate the
figures, is found in {\tt tutorial.m} in the {\tt examples/} directory.

\bfi % ffffffffffffffffffffffffffffffffffffffffffffffffffffffffffffffffffffff
a)\raisebox{-0.3\textwidth}{\ig{height=0.35\textwidth}{seg.eps}}
b)\raisebox{-0.3\textwidth}{\ig{height=0.35\textwidth}{dom.eps}}
\ca{a) circular closed segment, b) unit disc domain.
Both have a periodic trapezoidal quadrature rule
with $M=20$ quadrature points}{f:sd}
\efi

% ----------------------------------------------------------------------------
\section{Solving Laplace's equation in a disc}
\label{s:lap}

We start by setting up a domain in $\mathbb{R}^2$.
% in which a PDE will be solved.
Domains are built from segments which define their boundary.
To make the unit disc domain,
we first need a circle segment with center
0, radius 1, and angle range
$[0,2\pi)$, as follows,
\co{s = segment([], [0 1 0 2*pi])}
The object {\tt s} is indeed a circular segment, as we may check by
typing {\tt s.plot}, producing Fig.~\ref{f:sd}a.
%or by examining its contents by typing {\tt s}.
All segments have a {\em sense}, i.e.\ direction of travel:
for this segment it is counter-clockwise, as shown by the
downwards-pointing
arrow symbol overlayed onto the segment at about 9 o'clock.%
  \footnote{In fact, segments are parametrized internally as function $z(t)$
    of a real variable $t\in[0,1]$, and the sense is the direction of
    increasing $t$. Segment {\tt s} stores this function as {\tt s.Z}.}
Notice also normal vectors (short `hairs') pointing outwards
at each boundary point; our definition is that
normals on a segment always point to the {\em right} when traversing the
sense of the segment.

We create the domain interior to this segment with
\co{d = domain(s, +1)}
where the second argument (here $+1$, the only other option being $-1$)
specifies that the domain is to the `standard' side of the segment, which
we take to be such that the normals point {\em away from} the domain.
That is, with $+1$ the domain lies to the {\em left} of the segment
when traversed in its correct sense (with $-1$ the domain
would lie to the right of the segment.)
Typing {\tt d.plot} produces%
  \footnote{There are extra plotting options and features that
    are described in documentation such as {\tt help domain.plot}.
    E.g.\ in this figure a grid of points interior to the domain has been
    included, achieved with {\tt opts.gridinside=0.05; d.plot(opts);}
    %TODO: demo switch off corner fan.
  }
Fig.~\ref{f:sd}b.
Note that perimeter and area are automatically
labelled (these are only rough approximations intended for sanity checks).

\bfi % ffffffffffffffffffffffffffffffffffffffffffffffffffffffffffffffffffffff
a)\raisebox{-0.3\textwidth}{\ig{height=0.35\textwidth}{u.eps}}
b)\raisebox{-0.3\textwidth}{\ig{height=0.35\textwidth}{uerr.eps}}
\ca{a) Numerical solution field $u$, b) pointwise error $u-f$,
for Laplace's equation in the unit disc with $M=20$ quadrature points
and 8th-order harmonic polynomials.}{f:u}
\efi

Laplace's equation $\Delta u = 0$ is Helmholtz's equation with wavenumber
zero, which we set for this domain with,
\co{d.k = 0;}
Our philosophy is
to approximate the solution in the domain by a linear combination of
{\em basis functions}, each defined over the whole domain.
%We have to choose how many to use, i.e.\ the
%{\em order} of approximation.
We choose 8th-order harmonic polynomials
$u(z) = \sum_{n=0}^{8} c_n \,\mbox{Re}\,z^n +
\sum_{n=1}^{8} c_{-n}\,\mbox{Im}\,z^n$, where $\mbf{c}:=\{c_n\}_{n=-8}^{8}
\in\mathbb{R}^{17}$
is a coefficient vector, based at the origin 0,
%i.e.\ the sets $\{\mbox{Re } z^n\}_{n=0}^{10}$ and
%$\{\mbox{Im } z^n\}_{n=1}^{10}$,
using the command
\co{d.addregfbbasis(0, 8);}
Let's specify Dirichlet boundary data $f(z) = 
\ln |z-2-3i|$ for
$z$ on the segment%
  \footnote{In other words, $f(x,y) = \ln \sqrt{(x-2)^2+(y-3)^2}$
    for points $(x,y)$ on the boundary.}
by representing this as an anonymous function {\tt f}
and associating it with one side of the segment,
\begin{verbatim}
   f = @(z) log(abs(z-2-3i));
   s.setbc(-1, 'd', [], @(t) f(s.Z(t)));
\end{verbatim}
Note that we needed to pass in a function not of location $z$,
but of the segment parameter $t$;
this was achieved by
wrapping ${\tt f}$ around the parametrization function ${\tt s.Z}$.
The first argument $-1$ expresses that the boundary condition is to be
understood in the limit approaching from the side {\em opposite} the
segment's normal direction, which is where the domain is located.
The second argument {\tt 'd'} specifies that the data is Dirichlet.
 
Finally we use the domain to make a boundary-value
problem object {\tt p},
\co{p = bvp(d);}
and may then solve (in the least-squares sense)
a linear system for the coefficients
\co{p.solvecoeffs;}
If it is needed, {\tt p.co} now contains the coefficients vector $\mbf{c}$.
To evaluate and plot the solution we simply use,
\co{p.showsolution;}
The software chose an appropriate grid covering the domain
(points outside the domain are made transparent), giving Fig.~\ref{f:u}a.

\bfi % ffffffffffffffffffffffffffffffffffffffffffffffffffffffffffffffffffffff
a)\raisebox{-0.3\textwidth}{\ig{height=0.35\textwidth}{N.eps}}
b)\raisebox{-0.3\textwidth}{\ig{height=0.35\textwidth}{radfunc.eps}}
\ca{a) Convergence of boundary error $L^2$ norm for harmonic
polynomials for Laplace equation
in the unit disc, b) solution error for same boundary data $f$
in a smooth star-shaped `trefoil' domain (normals also shown).}{f:conv}
\efi

% ----------------------------------------------------------------------------
\section{Accuracy, convergence, and smooth domains}

How accurate was our numerical solution $u$? One measure is the
$L^2$ error on the boundary, and is estimated by
\co{p.bcresidualnorm}
which returns $2.09 \times 10^{-6}$.
However, since the function $f(z)$ is already harmonic in the domain,
it is in fact the unique solution, and we may plot the
pointwise error in $u$ by passing in the analytic solution as an option,
\co{opts.comparefunc = f; p.showsolution(opts);}
giving Fig.~\ref{f:u}b. Note that the color scale is $10^{-8}$.

In the above, boundary integrals were approximated using the default of
$M=20$ quadrature points, barely adequate given the
oscillatory error function in Fig.~\ref{f:u}b.
$M$ may be easily changed either by specifying
a non-empty first argument in the {\tt segment} constructor above, or
for an existing segment as follows,
\co{s.requadrature(50); p.solvecoeffs; p.bcresidualnorm}
which now gives $1.98\times 10^{-6}$, not much different than before.
Notice that we did not have to redefine the domain {\tt d} nor
the BVP object {\tt p}.

Exploring the convergence of the boundary error norm with the basis set order
needs a simple loop and figure,
\begin{verbatim}
   for N=1:15
     d.bas{1}.N = N; p.solvecoeffs; r(N) = p.bcresidualnorm;
   end
   figure; semilogy(r, '+-'); xlabel('N'); ylabel('bdry err norm');
\end{verbatim}
As Fig.~\ref{f:conv}a shows, the convergence is exponential.%
  \footnote{Asymptotically, error $\sim e^{-\alpha N}$. In fact the rate is
    $\alpha = \ln \sqrt{13}$, related to the conformal distance to
    the nearest singularity \cite{timothesis}, which here is at $2+3i$.}

Say we want to change the shape of segment {\tt s}, to
a smooth star-shaped `trefoil' domain expressed as by radius $R(\theta) =
1 + 0.3\cos 3\theta$ as a function of angle $0\le \theta< 2\pi$.
This is achieved by passing a 1-by-2
cell array containing the function $R$ and its
derivative $R' = dR/d\theta$ to a variant of the segment constructor,
\begin{verbatim}
   s = segment.radialfunc(50, {@(q) 1 + 0.3*cos(3*q), @(q) -0.9*sin(3*q)});
\end{verbatim}
We again chose $M=50$.
The analytic formula for $R'$ is needed to compute normal derivatives
to high accuracy.

One might ask: has this change to {\tt s} {\em propagated}
to the existing domain
object {\tt d} and BVP object {\tt p}, which both refer to it?
In contrast to the case of quadrature point number $M$ above,
the answer is no:
{\tt s} is overwritten by a newly-constructed object, while
{\tt d} and {\tt p} still contain handles pointing to the {\em old}
{\tt s}.
Furthermore, the fact that the segment had domain {\tt d}
attached to its `minus' or back side has been forgotten, as have the
boundary conditions.
(These segment properties are described in the \mpspack\ user manual.)
We must therefore rerun the code from Sec.~\ref{s:lap}
to construct {\tt d} and {\tt p} afresh, before solving.%
  \footnote{Note that in theory it would be possible to
    change one by one each of the segment properties, {\tt t}, {\tt w},
    {\tt speed}, etc, to define the new segment without changing its identity,
    but this is cumbersome. Similarly, searching and changing
    all references to a segment in the properties of {\tt d} and {\tt p}
    is cumbersome. Neither has been implemented since problem setup time is
    very rapid.}
The result, plotting the pointwise error as before,
is shown by Fig.~\ref{f:conv}b for $N=8$ and $M=50$.

The {\tt radialfunc} constructor above is limited to radial functions
with quadrature equidistant
in angle. Instead you may create a segment from arbitrary
smooth parametrizations $z(t)$ for $t \in[0,1]$, as long as $z'(t)$
is also given. For instance, a closed crescent-shaped analytic segment is
produced by 
\begin{verbatim}
   a = 0.2; b = 0.8; w = @(t) exp(2i*pi*t);
   s = segment(100, {@(t) w(t)-a./(w(t)+b), ...
                     @(t) 2i*pi*w(t).*(1 + a./(w(t) + b).^2)}, 'p');
\end{verbatim}
Note the nested anonymous functions for mathematical clarity.
Note also the new final argument {\tt 'p'} which enforces
periodic quadrature (the constructor doesn't try to guess your preferred rule).
In order to get high-order (or spectral) convergence, it is recommended
that you choose only smooth (or analytic) $z$.
If periodic quadrature is used,
this also applies to the 1-periodic extension of $z$ to the real line.
If $z(1)\neq z(0)$, the ends of the segment will not connect
up, and the domain constructor above will report an error.


% ----------------------------------------------------------------------------
\section{Helmholtz equation, exterior and multiply connected domains}

Changing from the Laplace to Helmholtz equation is as simple
as setting {\tt d.k} to a different value.
We start a fresh example: a exterior Helmholtz BVP
with Neumann boundary data, and the Sommerfeld radiation
condition \cite{coltonkress}. This has a unique solution.

The simplest unbounded domain is $\mathbb{R}^2$, which is created with
\co{d = domain();}
One may check that its area {\tt d.area} is $\infty$.
Exterior domains can be created by excluding a closed segment, for instance
the trefoil segment introduced above,
\begin{verbatim}
   tref = segment.radialfunc(50, {@(q) 1 + 0.3*cos(3*q), @(q) -0.9*sin(3*q)});
   d = domain([], [], tref, -1);                    % overwrites previous d
\end{verbatim}
Note the choice $-1$ for the direction argument, which states that
the domain lies on the `nonstandard' side of the segment, i.e.\
to the right side as the segment is traversed in its natural sense,
with the segment normals pointing {\em into} the exterior domain.





[TIMO insert mfsbasis, solve exterior smooth domain Helmholtz Dirichlet
BVP. Add your code to examples/tutorial.m]

\vspace{10ex}














\bfi % ffffffffffffffffffffffffffffffffffffffffffffffffffffffffffffffffffffff
a)\raisebox{-0.4\textwidth}{\ig{height=0.45\textwidth}{twoholes.eps}}
b)\raisebox{-0.4\textwidth}{\ig{height=0.45\textwidth}{tri.eps}}
\ca{a) A multiply-connected domain. b) A polygonal domain.}{f:doms}
\efi

A non-simply connected domain may be built by specifying excluded
regions from a simply connected bounded domain. For example,
to remove from an interior trefoil a circular `hole',
\begin{verbatim}
   tref.disconnect;                         % clears any domains from segment
   c = segment([], [0.5 0.4 0 2*pi]);       % new circular segment
   d = domain(tref, 1, c, -1);
\end{verbatim}
Note that segment {\tt tref} had previously
been `linked' to the old domain {\tt d}
at the start of this section, hence the need to `disconnect' it
(or create a fresh segment) before
building new domains from it. 
If the direction signs $+1$ and $-1$ are not correct as above, an
error is reported (check this!)
We may exclude two regions as follows, where the new region is a smaller copy
of the trefoil,
\begin{verbatim}
   tref.disconnect; c.disconnect;
   smtref = tref.scale(0.3);                % create new rescaled copy of tref
   smtref.translate(-0.3+0.4i);             % move the segment smtref
   d = domain(tref, 1, {c smtref}, {-1 -1});
\end{verbatim}
Typing {\tt d.plot} gives Fig.~\ref{f:doms}a. Notice that
the domain's boundary is the union of three segments. They are labeled
1, 2, and 3, showing the order in which segment handles
are stored internally in the domain object {\tt d.seg}.
The convention for plotting domains is that the normals
are those of the domain, rather than the normal intrinsic to each segment.
The figure shows all normals pointing away from the domain, as it should.
Similarly, the arrow directions are modified by the signs $(1,-1,-1)$ that
were passed in, so that, following the arrows the domain always lies to
the {\em left}. (The black semicircles will be explained in the next section.)


[MAYBE add mfsbasis example here with three src curves one for each segment?]

\vspace{10ex}




% ----------------------------------------------------------------------------
\section{Polygons, corners, and corner-adapted bases}
\label{s:poly}

So far each disconnected boundary piece of the domain
has been a single segment connected to itself head-to-tail.
More generally, a {\em list} of segments may be used to create 
these closed boundary pieces, as long as the last in the list
connects back to the starting point of the first.
For instance, a triangle is built from sending a list of
three line segments to the domain command;
this list may conveniently be made with a polygon constructor,
\begin{verbatim}
   s = segment.polyseglist([], [1, 1i, exp(4i*pi/3)]);
   tri = domain(s, 1);
\end{verbatim}
Since {\tt s} is a 1-by-3 array of (handles to) segments,
the 2nd argument is automatically vectorized to {\tt [1 1 1]}.
Each of the segments could have been made separately, e.g.\ the first by
\co{s(1) = segment([], [1 1i]);}
Fig.~\ref{f:doms}b shows {\tt tri.plot} output for this domain.%
  \footnote{Notice that periodic quadrature would now be inappropriate:
    in fact $M=20$ point Clenshaw-Curtis is used by default for open
    segments}

Let's discuss corners: their angles are indicated by the solid black
`fans' in the plot (before now these have had angle $\pi$, hence semicircles).
A black fan at the junction of two segments indicates that a corner
linkage was made (warnings will be given if any segment ends are left
dangling), and shows the angle range pointing {\em into} the domain.
%The data is stored in the domain properties as {\tt tri.
%and their start and end angles

Segment lists may also be sent to the excluded arguments of the
domain constructor, for instance to create the domain exterior
to the triangle,
\co{exttri = domain([], [], s(end:-1:1), -1);}
where it was important to reverse the order of the segment list,
since each was 
or to create the domain exterior to two nearby triangles,
\begin{verbatim}
   ss = s.translate(2);
   exttwotri = domain([], [], {s(end:-1:1), ss(end:-1:1)}, {-1, -1});
\end{verbatim}
We remind the reader that to create the above domains
using the existing segment array {\tt s}, each time a {\tt s.disconnect}
would be needed beforehand (this acts on all segments in the list).%
  \footnote{Also note that the domains previously linked to the segments, such
    as {\tt tri}, would be left `dangling' since {\tt s} no longer is linked
    back to them. Attempts to use {\tt tri} in a BVP would now be doomed
    unless the data {\tt s(:).dom} were manually rewritten to point to
    the domain (see manual). When in doubt, disconnect segments then remake
    all domains.}
A universal rule is:
\begin{center}\framebox{\mbox{Each side of a segment may be associated with
at most one domain}}\end{center}

\bfi % ffffffffffffffffffffffffffffffffffffffffffffffffffffffffffffffffffffff
a)\raisebox{-0.35\textwidth}{\ig{height=0.4\textwidth}{triFB.eps}}
b)\raisebox{-0.35\textwidth}{\ig{height=0.4\textwidth}{triu.eps}}
\ca{a) Comparing convergence of boundary error norm for Fourier Bessel
vs corner-adapted fractional-order Fourier Bessel basis sets
in a triangle with unity Dirichlet boundary data, for Helmholtz
BVP with $k=10$, on log-log axes. b) The solution function.}{f:triconv}
\efi

Say we want to solve an interior Helmholtz BVP on the original triangle
of this section, and {\tt s} and {\tt tri} have been set up as with the
first two commands of this section but with $M=50$ quadrature points.
Say we want constant boundary data 1.
We may use a Fourier-Bessel basis set, solve, and
plot error convergence with a simple code,
\begin{verbatim}
   s.setbc(-1, 'd', [], @(t) 1);             % note inline "1" function
   tri.addregfbbasis(0, []); tri.bas{1}.rescale_rad = 1.0; % for stability
   p = bvp(tri);
   tri.k = 10;                               % set wavenumber
   Ns = 2:2:40; for i=1:numel(Ns)
     tri.bas{1}.N = Ns(i); p.solvecoeffs; r(i) = p.bcresidualnorm;
   end
   figure; loglog(2*Ns, r, '+-'); xlabel('# dofs'); ylabel('bdry err norm');
\end{verbatim}
This produces the algebraic convergence%
  \footnote{The power law for convergence is related to the largest corner
angles and is discussed in \cite{Ei74}.}
shown in blue in Fig.~\ref{f:triconv}a.

We now show how corner-adapted basis sets may be substituted for
the above basis set to achieve superior convergence and hence
more efficiency. We clear the previous basis set from the domain
and add fractional-order (i.e. `wedge' expansion) Fourier Bessel bases
at two of the corners, of both cos and sin type.%
  \footnote{Since only one of the corners is
    singular, it is possible to omit a corner
    expansion from one of the other corners, for instance the 3rd one.
    This done with the options {\tt opts.cornerflags = [1 1 0];}.}
Repeating the convergence
study and comparing against the previous data is easy,
%super-algebraic ?
\begin{verbatim}
tri.clearbases; opts.rescale_rad = 1.9; opts.cornerflags = [1 1 0];
tri.addcornerbases([], opts);
Ns = 1:20; for i=1:numel(Ns)
  for j=1:3, tri.bas{j}.N=Ns(i); end            % sets each order to N
  p.solvecoeffs; r(i) = p.bcresidualnorm;
end
hold on; loglog(4*Ns, r, 'r+-');                % total # dofs is 4N
\end{verbatim}
The new convergence data is shown in red in Fig.~\ref{f:triconv}a;
initially it is superalgebraic.
[WHY IS IT NOT SUPER-ALGEBRAIC, TIMO? IT IS ALSO too SENSITIVE to rescalerad - WHY?]
The solution field is Fig.~\ref{f:triconv}b, and its large values
shows that we are close to a Dirichlet resonance of the domain.
Since only one of the corners is singular, it is possible to omit a corner
expansion from one of the other corners, for instance the 3rd one, by
using {\tt opts.cornerflags = [1 1 0];}.
Finally the basis sets geometry in the domain can be visualized by
{\tt tri.showbasesgeom;} which shows the wedges implied by the corner
expansions.


% ----------------------------------------------------------------------------
\section{Scattering and transmission problems}

{\tt scattering} class.

Incident wave.

(change showsolution to handle air vs non-air domains).

Refractive indices, overall wavenumber.

% ----------------------------------------------------------------------------
\section{Layer potentials}




\bibliographystyle{siam} 
\bibliography{alexrefs}
\end{document}
