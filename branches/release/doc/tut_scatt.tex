\bfi % ffffffffffffffffffffffffffffffffffffffffffffffffffffffffffffffffffffff
\ig{width=\textwidth}{soft.eps}
\ca{Sound-soft (homogeneous Dirichlet BC) scattering from a smooth
trefoil at $k=30$ using an MFS basis.
Left: incident wave $u_{inc}$. Center: scattered wave
$u_s$. Right: their sum, the total solution field $u$.}{f:soft}
\efi



% ----------------------------------------------------------------------------
\section{Scattering and transmission problems}
\label{s:scatt}

Scattering of scalar waves (acoustic, or suitably-polarized Maxwell
equations) involves finding a Helmholtz solution $u = u_{inc} + u_s$
with homogeneous boundary conditions
on an obstacle, given an incident (often plane-wave)
Helmholtz solution $u_{inc}$.
The unknown $u_s$ is then the solution to the radiative
exterior BVP solved in Sec.~\ref{s:ext}
(see Sec.~\ref{s:scatbvp} for more detail), with inhomogenous boundary data
related to $u_{inc}$.

We designed a {\tt scattering} class to 
implement this in a user-friendly fashion. Creating the analytic
{\tt tref} segment (with $M=250$)
and its exterior domain {\tt d} as in the start of
Sec.~\ref{s:ext}, we set up and solve the sound-soft scattering
problem as follows,
\begin{verbatim}
opts.tau = 0.05; d.addmfsbasis(tref, 200, opts);  % basis set for ext domain
s.setbc(1, 'D', []);                  % homogeneous Dirichlet BCs: sound-soft
p = scattering(d, []);
k = 30; p.setoverallwavenumber(k);
p.setincidentwave(pi/6);              % incident plane wave with given angle
p.solvecoeffs;                        % fills matrices, least-squares soln
\end{verbatim}
The quadrature approximation to the $L^2$ boundary error is $3\times10^{-10}$
as given by {\tt p.bcresidualnorm}; exponential convergence could be checked
with a little loop as in Sec.~\ref{s:conv}. 
The incident wave, scattered wave, and solution are plotted%
  \footnote{We in fact tweaked the grid spacing and bounding-box options via
{\tt o.dx=0.01; o.bb = 2*[-1 1 -1 1]; p.showthreefields(o);}}
as in Fig.~\ref{f:soft} by typing {\tt p.showthreefields}.
The solution took around 0.1 sec; the time for plotting is around 1 sec, but
depends on the grid resolution requested.
A good choice of the {\tt opts.tau} distance parameter must usually found
by trial and error, examining the charge point locations and solution quality.

We may trivially switch to other physical boundary conditions by replacing
the {\tt setbc} command with either of the following,
\begin{verbatim}
s.setbc(1, 'N', []);                  % homogeneous Neumann: sound-hard
s.setbc(1, 1i*k, 1);                  % homogeneous Robin: impedance
\end{verbatim}
In the latter case a boundary condition $iku + u_n=0$ is applied, which
strongly absorbs incident waves.

\bfi % ffffffffffffffffffffffffffffffffffffffffffffffffffffffffffffffffffffff
\ig{width=\textwidth}{diel.eps}
\ca{Transmission scattering from a smooth
trefoil with refractive index (interior to exterior wavenumber ratio) 1.5
and exterior wavenumber $k=30$,
using MFS bases inside and out.
Left: incident wave $u_{inc}$. Center: scattered wave
$u_s$. Right: total solution field $u$.}{f:diel}
\efi

A transmission problem involves coupled regions of different wavenumber,
for instance wavenumber $k$ in an exterior domain, and wavenumber $nk$ in
its complement, where $n$ is the `refractive index'%
  \footnote{Here we borrow notation from the optical application.}
of the interior.
Reusing the above domain {\tt d} and its basis, we need a new interior domain
{\tt di}
and its MFS basis (with charge points this time outside the boundary curve),
\begin{verbatim}
di = domain(tref, 1); di.refr_ind = 1.5;
opts.tau = -0.03; di.addmfsbasis(tref, 220, opts);
tref.setmatch('diel', 'TM');
p = scattering(d, di);               % d is an `air' domain, di not air domain
p.setoverallwavenumber(k);
\end{verbatim}
The matching condition that $u$ and $u_n$ are continuous across the interface
is appropriate for TM-polarized Maxwell (in two dimensions, i.e.\
$z$-invariant), for a dielectric problem, hence the {\tt setmatch}
arguments.
%This is also appropriate for acoustics when $u$ represents
%the pressure field. - NOTE SURE ABOUT THIS.
Then, {\tt p.solvecoeffs; p.showthreefields;} produces
Fig.~\ref{f:diel}, which has error in boundary matching
(summing the $L^2$ errors in the jump in $u$ and jump in $u_n$)
of $10^{-6}$ given by {\tt p.bcresidualnorm}.
It can be verified that the convergence is exponential; MFS bases are an
excellent choice for analytic boundaries.%
  \footnote{If condition number is not a problem \cite{mfs}.
    For well-conditioned
    formulations, layer potentials are needed as described in the next
    section.}

Notice that the {\tt scattering} constructor takes {\em two} arguments
(recall {\tt bvp} took only one, the domain or list of domains in the problem).
The first argument is the (list of)
`air' (as opposed to dielectric) domains, the second the `non-air' domains.
The former are interpreted as the domains which have
the incident plane wave as their $u_{inc}$;
notice the non-air domain {\tt di} has trivial $u_{inc}$ in the left plot
of Fig.~\ref{f:diel}, because this index $n\neq 1$ domain
does not have the incident plane wave as a Helmholtz solution of the
correct wavenumber $nk$.
In the above, there was no freedom of choice of the air/non-air categorization,
but in more complicated problems the exterior region may be divided into
subdomains, some of which are passed in as air domains, others not;
see Sec.~\ref{s:square}.
This enables dielectric-coated metals, dielectrics with interior holes,
etc, to be solved; see Sec.~\ref{s:dielhole}.
