% ----------------------------------------------------------------------------
\section{Layer potentials}

Layer potentials are representations of Helmholtz solutions
involving an integral over a surface, i.e. a boundary segment
\cite{coltonkress}.
They are similar the MFS representations discussed above, with the
crucial advantage that a {\em second-kind}
formulation is often possible, i.e. the operator involved is
identity plus compact, and the resulting matrix problems are well-conditioned.
It is easy to set up in a domain a single-layer potential (SLP) density
\be
u(\bx) = {\cal S}\sigma := \int_\Gamma \Phi(\bx-\by) \sigma(\by) ds_\by
\ee
where $ds$ is arclength, or double-layer potential (DLP) density
\be
u(\bx) = {\cal D}\tau := \int_\Gamma \frac{\partial \Phi(\bx-\by)}{\partial n_\by} \tau(\by) ds_\by
\ee
where $\Gamma$ is a segment, and $\sigma$ and $\tau$ are functions on $\Gamma$.
For example, using as $\Gamma$ the
trefoil segment {\tt tref} defined in Sec.~\ref{s:ext}, a DLP is set up
by
\begin{verbatim}
  d = domain([], [], tref, -1); d.k = 10;    % external domain, wavenumber k
  d.addlayerpot([], 'd');                    % adds DLP to bdry of d
\end{verbatim}

The coefficients in {\tt p.co} represent density function values
at the quadrature points.

TO FINISH.


