\appendix

\setlength{\parskip}{\baselineskip}
\setlength{\parindent}{0pt}

\section{Class Descriptions}


\paragraph{\tt POINTSET}

Create a pointset object with locations and normal vectors as
complex numbers.

A pointset is simple object containing a list of points in 2D, plus possibly
associated normal directions.  It is used to store quadrature points on a
segment, and also evaluation point lists. Coordinates are stored as 
complex numbers.

\textbf{Constructors}

{\tt p = POINTSET()}
creates an empty object.
 
{\tt p = POINTSET(x)} where x is m-by-1 array, creates pointset with m points, where
the ith point has Cartesian coordinates (Re x(i), Im x(i)).
  
{\tt p = POINTSET(x, nx)} where x is above and nx has same size as x, creates
pointset with coordinates x (interpreted as above) and associated normals
nx (interpreted in the same way). The Euclidean lengths of the vectors in
nx are not required to be, nor changed to, unity.

See also: {\tt POINTSET/plot}, {\tt SEGMENT} which builds on {\tt POINTSET}

\textbf{Methods}

none


\newpage

\paragraph{\tt SEGMENT} create segment object

\textbf{Constructors}

{\tt s = SEGMENT(M, [xi xf])}  creates a line segment object from xi to xf, both
complex numbers.

{\tt s = SEGMENT(M, [xc R ti tf])} creates a circular arc segment with center
xi, radius R and angles from ti to tf. The order is important. If tf>ti
the orientation is counter-clockwise, otherwise clockwise.
  
{\tt s = SEGMENT(M, {Z, Zp})} creats an analytic curve given by the image of 
the analytic function Z:[0,1]->C. Zp must be the derivative of Z. 
Z and Zp are function handles. 
  
{\tt s = SEGMENT(M, {Z, Zp, Zpp})} works as above but also takes the second
derivative Z'' of Z. This is useful for layer potentials.
 
{\tt s = SEGMENT(M, p, qtype)} where p is any of the above, chooses quadrature type\\
    {\tt qtype = 'p'}: periodic trapezoid (appropriate for periodic segments, M pts)\\
          {\tt 't'}: trapezoid rule (ie, half each endpoint, M+1 pts)\\
           {\tt 'c'}: Clenshaw-Curtis (includes endpoints, M+1 pts)\\
           {\tt 'g'}: Gauss (takes $O(M^3)$ to compute, M pts)\\

If M is empty, a default value of 20 is used.

{\tt s = SEGMENT()} creates an empty segment object.

See also: {\tt POINTSET}, {\tt segment/PLOT}

\textbf{Methods}

{\tt requadrature(segs, M, qtype)} change a segment's quadrature
scheme or number of points

{\tt setbc(seg, pm, a, b, f)} set a (in)homogeneous boundary
condition on a segment

{\tt setmatch(seg, a, b, f, g)} set (in)homogeneous matching
conditions across a segment 

{\tt newseg = scale(seg, fac)} rescale (dilate) a segment (or list)
about the origin

{\tt newseg = translate(seg, a)} translate a segment (or list of segments)

{\tt newseg = rotate(seg, t)} rotate a segment (or list of
segments) about the origin

{\tt newseg = reflect(seg, ax)} reflect a segment (or list of
segments) about x or y axis

{\tt disconnect(segs)} disconnect a segment or segment list from any domains

{\tt h = plot(s, pm, o)} plots a directed segment on current
figure, using its quadrature pts


{\tt s = polyseglist(M, p, qtype)}\\
{\tt s = polyseglist(M, p)}\\
(Static Method) create closed list of segment objects from CCW polygon vertices

{\tt s = radialfunc(M, fs)} (Static Method) closed segment from
radial function r=f(theta) and derivatives

{\tt s = smoothstar(M, a, w)} (Static Method) single-freq
oscillatory radial function closed segment

{\tt [a b] = dielectriccoeffs(pol, np, nm)} (Static Method) give a
and b coeff pairs (1-by-2) from refractive indices


 
\newpage

\paragraph{\tt DOMAIN} create an interior/exterior domain possibly with
excluded subregions

A domain is an ordered connected list of segments defining the exterior
boundary, with zero or more ordered connected lists of segments defining the
boundaries of any interior excluded regions. If the exterior boundary is empty
the domain from which interior regions are possibly excluded is taken to be
the whole plane, resulting in an unbounded exterior domain.

\textbf{Constructors}

{\tt d = DOMAIN(s, pm)} creates an interior domain whose boundary is the list
of handles of segment objects s, using the list of senses pm (each element
is the number +1 or -1). A warning is given if the segments do not appear
to connect up at corners. Normals should all point outwards, ie away from
the domain, otherwise a warning is given. If pm has only 1 element, it will
be duplicated as necessary to match the size of s.

{\tt d = DOMAIN()}\\
{\tt d = DOMAIN([], [])}\\
creates an exterior domain equal to the
whole plane $\mathbb{R}^2$.

{\tt d = DOMAIN([], [], si, pmi)} creates an exterior domain equal to the whole
plane minus an excluded region whose (non-intersecting) boundary is given
by segment list si and sign list pmi (which have the same format as s, pm
above). If si and pmi are instead cell arrays of segment lists and
corresponding sign lists, each cell element is an excluded region. As
above, warnings are given if intersections or incorrect normals are found.

{\tt d = DOMAIN(s, pm, si, pmi)} combines the above features, creating a bounded
domain with excluded region(s).

See also: {\tt SEGMENT}, {\tt domain/PLOT}

\textbf{Methods}

{\tt norout = normalscheck(d)} check that senses of normals point
away from a domain

{\tt i = inside(d, p)} return true (false) for points inside
(outside) a domain

{\tt x = x(d)} return column vector of quadrature points on a domain boundary

{\tt nx = nx(d)} return column vector of unit outward normals on a
domain boundary

{\tt w = w(d)} return row vector of quadrature weights for a domain
boundary

{\tt bb = boundingbox(d)} return bounding [xmin xmax ymin ymax] for
interior domain

{\tt xc = center(d)} return center x (as complex number) of bounding
box of domain 

{\tt diam = diam(d)} approximate diameter of an interior domain

{\tt [zz ii gx gy] = grid(d, dx)} make grid covering interior
domain, or some of exterior domain

{\tt deletecorner(d,j)} remove a given corner number from a domain

{\tt Nf = Nf(d)} total number of basis functions (dofs) associated
with domain

{\tt clearbases(d)} remove all basis set associations from a domain

{\tt showbasesgeom(d)} show geometry of basis objs

{\tt addconnectedsegs(d, s, pm, o)} append a domain's params,
corners given conn seg list

{\tt addregfbbasis(d, origin, N, opts)} create a regular
Fourier-Bessel basis object in a domain

{\tt addnufbbasis(d,origin,nu,offset,branch,N,opts)} create a
fractional-order Fourier-Bessel basis set

{\tt addcornerbases(d, N, opts)} add irreg Fourier-Bessel basis to
each corner of a domain

{\tt addrpwbasis(d, N, opts)} create a real plane wave basis object
in a domain

{\tt addmfsbasis(d, varargin} Add an MFS (fundamental solutions)
basis to a domain

{\tt b = addlayerpot(d, a, segs, opts)} create a layer-potential
basis set object in a domain

{\tt [A A1 A2] = evalbases(d, p, opts)} evaluate all basis sets in
a domain object, on a pointset

{\tt h = plot(d, o)} show domain (or list of domains) on current figure

{\tt setrefractiveindex(doms, n)} sets n for a list of domains

{\tt v = approxpolygon(seg, pm)} (Static Method) stack together
approximating polygon vertices of segment list

{\tt h = showsegments(seg, pm, o)} plot signed segment list to
current figure (domain helper)

{\tt x = stackquadpts(seg, pm)} helper routine, ordered quad pts
from signed connected seg list 

{\tt h = showdomains(dlist, opts)} plot all domains on current
figure using a color for each

\newpage

\paragraph{\tt BASIS (ABSTRACT)} Abstract class that defines the
interfaces which are common for all basis objects


\textbf{Methods}

{\tt [A, A1, A2] = eval(b, pts)} (abstract) evaluate a basis on a set of points

{\tt Nf = Nf(b, opts)} (abstract) method returning number of degrees of freedom

{\tt updateN(b,N)} set the number of basis functions

{\tt k = k(b, opts)} looks up the basis wavenumber





\newpage 

\paragraph{\tt MFSBASIS : BASIS} create a fundamental solutions (MFS) basis
set.

\textbf{Constructors}

{\tt b = MFSBASIS(Z, N, opts)} creates an MFS basis using charge points
    at Z(t), where t are equally distributed in [0,1]. Z is a handle
   to a function. If the charge curve is to be closed, then Z must be
   1-periodic. N may be empty.

{\tt b = MFSBASIS({Z, Zp}, N, opts)} is as above. But Zp additionally specifies
   the derivative of Z. This is necessary if dipole-type charges (double
   layer potentials) are used. N may be empty.

{\tt b = MFSBASIS(p, opts)} describes MFS charge points in terms of a point
   set object p. If double layer potentials are used then p also needs to
   contain normal directions.

{\tt b = MFSBASIS(s, N, opts)} uses the function handles from segment s to choose
   N charge points. It is equivalent to b = MFSBASIS({s.Z, s.Zp}, N, opts).

  In each of the above, opts is an optional structure with optional fields:\\
  {\tt opts.eta}   - (inf) If eta=inf use single layer potential MFS basis. For
               eta $\neq$ inf use linear combination of double and single layer
               potential MFS basis\\
  {\tt opts.fast}  - (0) Hankel evaluation method (0=matlab; 1,2=faster)\\
  {\tt opts.real}  - (false) If true, use real part of Hankel functions only\\
  {\tt opts.tau}   - (0) Creates charge curve Z(t+i.tau) rather than Z(t)\\

\textbf{Methods}

{\tt showgeom(bas, opts)} plots geometry of MFS basis charge points on current figure


\newpage

\paragraph{\tt NUFBBASIS : BASIS} create a fractional-order Fourier-Bessel
basis set

\textbf{Constructors}


{\tt b = NUFBBASIS(origin, nu, offset, branch, N)} creates a basis of
fractional- 
   order Fourier-Bessel functions appropriate for expansion of the Helmholtz
   equation in a wedge of angle pi/nu > 0. The orders are nu*(1:N) (for sine
   angular functions) or nu*(0:N) (for cosine angular functions). The other
   arguments are:\\ 
     {\tt origin}: The origin of the Fourier-Bessel functions (as complex number)\\
     {\tt offset}: The direction that corresponds to the angular variable theta=0\\
     {\tt branch}: Direction of the branch cut (point on the unit circle);\\
             it must not point into the affected domain.\\
     {\tt N}     : degree of the basis set

   The wedge lies counterclockwise from offset, spanning pi/nu in angle.
   As with all basis types, the wavenumber k is determined by that of the
   affected domain. k=0 gives fractional-power Laplace equation solutions.

 {\tt b = NUFBBASIS(origin, nu, offset, branch, N, opts)} as above but permits
   optional arguments to be selected. Currently supported is\\
         {\tt opts.type}:\\    's'  Create basis of Fourier-Bessel sine fct.\\
                          'c'  Create basis of Fourier-Bessel cosine fct.\\
                          'cs' Create basis of Fourier-Bessel cosine and
                               sine functions (default)\\
   The reason for having 'cs' rather than using separate 's' and 'c' objects
   is a factor of 2 in speed: the set of Bessel evaluations is reused.

See also: DOMAIN/ADDNUFBBASIS

\textbf{Methods}

{\tt h = showgeom(nufb, opts)} 
show corner wedge location, geometry and branch cut

{\tt sc = Jrescalefactors(nufb, n)} given list of orders, return FB
J-rescaling factors


\newpage

\paragraph{\tt REGFBBASIS : BASIS} create a regular Fourier-Bessel basis set

\textbf{Constructors}

{\tt b = REGFBBASIS(origin, N)} creates a regular Fourier-Bessel basis object
   with given origin, and order N. As with all basis types, the wavenumber
   k is determined by that of the affected domain, as follows:

   $k=0$ gives harmonic polynomials,\\
       $\{1, \text{Re} z, \text{Re} z^2, ..., \text{Re} z^N, \text{Im} z, ..., \text{Im} z^N\}$ (real case)\\
    or $\{z^{-N}, z^{-N+1}, ..., z^N\}$ (complex case)\\
  $k>0$ gives generalized harmonic polynomials,
       $\{J_n(kr)\cos(n\theta)\}$ $n=0,..,N$ and
       $\{J_n(kr)\sin(n\theta)\}$ $n=1,..,N$ (real case)\\
    or $\{J_n(kr)\exp(1in\theta)\}$ $n=-N,..,N$ (complex case)

  {\tt b = REGFBBASIS(origin, N, opts)} is as above, with options:\\
   {\tt opts.real}: if true (default), use real values (cos, sin
   type), otherwise use complex exponentials with orders -N through N.\\
  {\tt opts.rescale\_rad}: if positive, rescales the basis coefficients i.e.
              columns of evaluation matrix, such that the value at radius
              rescale\_rad is O(1). (default is 0, giving no rescaling)\\
   {\tt opts.besselcode}: math library to use for J Bessel evaluation ($k>0$)\\
              {\tt 'r'} use downwards-recurrence in Matlab, fast, but
                       relative accuracy not guaranteed.\\
              {\tt 'm'} use Matlab's built-in besselj, is slower (default).\\
              {\tt 'g'} use GNU Scientific Library via MEX interface, fast.

See also: {\tt DOMAIN/ADDREGFBBASIS}

\textbf{Methods}

{\tt showgeom(regfb, opts)} plot regular FB basis set geometry info
(just origin now)

{\tt sc = Jrescalefactors(regfb, n)} given list of orders, return FB
J-rescaling factors

\newpage

\paragraph{\tt RPWBASIS : BASIS} create a real (as opposed to evanescent)
plane wave basis set

\textbf{Constructors}

{\tt b = RPWBASIS(N)} creates a real plane wave basis object, with N directions
   $\theta_j$ equally spaced in $(0,\pi]$, i.e. $\theta_j = \pi j/N$. As with all basis
   types, the wavenumber k is determined by that of the affected domain.
   Basis functions are:\\
      $\{\cos(k n_j z)\}$ $j=1,..,N$ and $\{\sin(k n_j.z)\}$ $j=1,..,N$       (real case)\\
   $\{\exp(i k n_j z)\}$ $j=1,..,N$ and $\{\exp(-i k n_j z)\}$ $j=1,..,N$  (complex case)\\
     where the direction unit vectors are $n_j = (\cos\theta_j, \sin\theta_j)$

   If k=0 a warning is produced since this basis does not have a useful
   limiting form as k tends to zero (regfbbasis should be used instead).

  {\tt b = RPWBASIS(N, opts)} does the same, except allowing user options:\\
   {\tt opts.real}: if true, real case (cos/sin type), otherwise complex case.

See also: {\tt DOMAIN/ADDRPWBASIS}


\textbf{Methods}

{\tt showgeom(b, opts)} plot real plane wave basis set geometry information

\newpage

\paragraph{\tt LAYERPOT : BASIS} create a layer potential basis set on a segment

\textbf{Constructors}

{\tt b = LAYERPOT(seg, a, opts)} where {\tt a = 'single'} or {\tt
  'double'} creates a layer
   potential basis object with density on segment seg, with options:\\
    {\tt opts.real}: if true, real valued ($Y_0$), otherwise complex
    ($H_0$ outgoing).\\
    {\tt opts.fast}: if 0 use matlab Hankel, 1 use fast fortran
    Hankel, 2 faster.\\
   If a is instead a 1-by-2 array, then a mixture of a(1) single plus a(2)
   double is created, useful for Brakhage, Werner, Leis \& Panich type
   combined representations. As usual for basis sets, the wavenumber k is
   determined by that of the affected domain.

   Note that the discretization of the layerpot is given by that of the seg,
    apart from periodic segments where new quadrature weights may be used.

See also: {\tt DOMAIN/ADDLAYERPOT},

\textbf{Methods}

{\tt showgeom(b, opts)} show discrete points of segments


{\tt [A Sker] = S(k, s, t, o)} (Static Method) double layer potential
discretization matrix for density on a segment

{\tt [A Dker\_noang cosker] = D(k, s, t, o)} (Static Method) double
layer potential discretization matrix for density on a segment

{\tt [A Sker Dker\_noang] = T(k, s, t, o)} (Static Method) deriv of double layer
potential discr matrix for density on a segment

{\tt A = localfromSLP(s, Jexp)} (Static Method) return matrix taking SLP density
values on seg to J (local) exp

{\tt A = localfromDLP(s, Jexp)} (Static Method) return matrix taking DLP density
values on seg to J (local) exp

{\tt A = fundsol(r, k)} (Static Method) return matrix of fundamental
solutions

{\tt [B radderivs] = fundsol\_deriv(r, cosphi, k, radderivs)} (Static Method) return
matrix of normal derivatives of fundamental solutions 

\newpage

\paragraph{\tt QUADR} static class of quadrature rules

\textbf{Methods}

{\tt [x w] = peritrap(N)} (Static Method) trapezoid quadrature rule for periodic
functions

{\tt [x w] = traprule(N)} (Static Method) quadrature points and weights for composite
N+1-point trapezoid rule

{\tt [x,w] = gauss(N)} (Static Method) nodes x (Legendre points) and weights w 
for Gauss quadrature

{\tt [x w] = clencurt(N)} (Static Method) nodes x (Chebyshev points) and weights
w for Clenshaw-Curtis quadrature

{\tt [x,w,cs,ier]=kapurtrap(n,m)} (Static Method) nodes and weights of n-point
corrected trapezoidal quadrature formula on [0,1]

{\tt Rjn = kress\_Rjn(n)} (Static Method) Kress weights

{\tt D = perispecdiffrow(N)} (Static Method) row 1 of N-pt spectral
$2\pi$-periodic differentiation matrix

\newpage

\paragraph{\tt PROBLEM (ABSTRACT)} abstract class defining interfaces for
Helmholtz/Laplace BVPs

\textbf{Methods}

{\tt fillquadwei(pr)} compute sqrt of bdry quadrature weights vector
for a problem

{\tt [N noff] = setupbasisdofs(pr)} set up indices of all basis
degrees of freedom in problem

{\tt A = fillbcmatrix(pr, opts)} computes matrix mapping basis coeffs
to bdry mismatch

{\tt [u di] = pointsolution(pr, p)} evaluate solution to a problem on
a pointset, given coeffs

{\tt [u gx gy di] = gridsolution(pr, o)} evaluate solution to a
problem over a grid, given coeffs

{\tt o = gridboundingbox(pr, o)} set default options giving grid spacing
and bound box

{\tt h = showbdry(pr)} shows boundary segments in a problem, with
their natural sense

{\tt h = showbasesgeom(pr)} plot geometry of all basis objects in a problem object

{\tt h = plot(pr)} show all geometry information in a problem object

{\tt setoverallwavenumber(pr, k)} set problem k, in each domain using
refactive index

{\tt updateN(pr,N)} Update the number of basis functions in each basis set

{\tt [u gx gy di h] = showsolution(pr, o)} plot figure with solution Re u
over all domains in problem

\newpage




 
 
 
















%%% Local Variables: 
%%% mode: latex
%%% TeX-master: "manual"
%%% End: 
