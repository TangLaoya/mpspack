\appendix

\setlength{\parskip}{\baselineskip}
\setlength{\parindent}{0pt}

\section{Class Descriptions}


\paragraph{POINTSET}

Create a pointset object with locations and normal vectors as
complex numbers.

A pointset is simple object containing a list of points in 2D, plus possibly
associated normal directions.  It is used to store quadrature points on a
segment, and also evaluation point lists. Coordinates are stored as 
complex numbers.

\textbf{Constructors}

\textbf{p = POINTSET()}
creates an empty object.
 
\textbf{p = POINTSET(x)} where x is m-by-1 array, creates pointset with m points, where
the ith point has Cartesian coordinates (Re x(i), Im x(i)).
  
\textbf{p = POINTSET(x, nx)} where x is above and nx has same size as x, creates
pointset with coordinates x (interpreted as above) and associated normals
nx (interpreted in the same way). The Euclidean lengths of the vectors in
nx are not required to be, nor changed to, unity.

\textbf{Methods}

none

See also: POINTSET/plot, SEGMENT which builds on POINTSET


%%% Local Variables: 
%%% mode: latex
%%% TeX-master: "manual"
%%% End: 
