% ----------------------------------------------------------------------------
\section{Helmholtz equation, exterior and multiply connected domains}

Changing from the Laplace to Helmholtz equation is as simple
as setting {\tt d.k} to a positive value.
We start a fresh example: a exterior Helmholtz BVP
with Neumann boundary data, and the Sommerfeld radiation
condition \cite{coltonkress}. This has a unique solution.

The simplest unbounded domain is $\mathbb{R}^2$, which is created with
\co{d = domain();}
One may check that its area {\tt d.area} is $\infty$.
Exterior domains can be created by excluding a closed segment, for instance
the trefoil segment introduced above,
\begin{verbatim}
   tref = segment.radialfunc(50, {@(q) 1 + 0.3*cos(3*q), @(q) -0.9*sin(3*q)});
   d = domain([], [], tref, -1);                    % overwrites previous d
\end{verbatim}
Note the choice $-1$ for the direction argument, which states that
the domain lies on the `nonstandard' side of the segment, i.e.\
to the right side as the segment is traversed in its natural sense,
with the segment normals pointing {\em into} the exterior domain.






[TIMO insert mfsbasis, solve exterior smooth domain Helmholtz Dirichlet
BVP. Add your code to examples/tutorial.m]

\vspace{10ex}














\bfi % ffffffffffffffffffffffffffffffffffffffffffffffffffffffffffffffffffffff
a)\raisebox{-0.4\textwidth}{\ig{height=0.45\textwidth}{twoholes.eps}}
b)\raisebox{-0.4\textwidth}{\ig{height=0.45\textwidth}{tri.eps}}
\ca{a) A multiply-connected domain. b) A polygonal domain.}{f:doms}
\efi

A non-simply connected domain may be built by specifying excluded
regions from a simply connected bounded domain. For example,
to remove from an interior trefoil a circular `hole',
\begin{verbatim}
   tref.disconnect;                         % clears any domains from segment
   c = segment([], [0.5 0.4 0 2*pi]);       % new circular segment
   d = domain(tref, 1, c, -1);
\end{verbatim}
Note that segment {\tt tref} had previously
been `linked' to the old domain {\tt d}
at the start of this section, hence the need to `disconnect' it
(or create a fresh segment) before
building new domains from it. 
If the direction signs $+1$ and $-1$ are not correct as above, an
error is reported (check this!)
We may exclude two regions as follows, where the new region is a smaller copy
of the trefoil,
\begin{verbatim}
   tref.disconnect; c.disconnect;
   smtref = tref.scale(0.3);                % create new rescaled copy of tref
   smtref.translate(-0.3+0.4i);             % move the segment smtref
   d = domain(tref, 1, {c smtref}, {-1 -1});
\end{verbatim}
Typing {\tt d.plot} gives Fig.~\ref{f:doms}a. Notice that
the domain's boundary is the union of three segments. They are labeled
1, 2, and 3, showing the order in which segment handles
are stored internally in the domain object {\tt d.seg}.
The convention for plotting domains is that the normals
are those of the domain, rather than the normal intrinsic to each segment.
The figure shows all normals pointing away from the domain, as it should.
Similarly, the arrow directions are modified by the signs $(1,-1,-1)$ that
were passed in, so that, following the arrows the domain always lies to
the {\em left}. (The black semicircles will be explained in the next section.)


[MAYBE add mfsbasis example here with three src curves one for each segment?]

\vspace{10ex}


