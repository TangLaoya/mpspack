\documentclass[11pt]{article}
\usepackage{graphicx,bm,amssymb,amsmath,amsthm}
% timo macros...
\newcommand{\bx}{\textbf{x}}
\newcommand{\by}{\textbf{y}}
\newcommand{\bs}{\textbf{s}}

% alex macros...
\newcommand{\bi}{\begin{itemize}}
\newcommand{\ei}{\end{itemize}}
\newcommand{\ben}{\begin{enumerate}}
\newcommand{\een}{\end{enumerate}}
\newcommand{\be}{\begin{equation}}
\newcommand{\ee}{\end{equation}}
\newcommand{\bea}{\begin{eqnarray}}
\newcommand{\eea}{\end{eqnarray}}
\newcommand{\bc}{\begin{center}}
\newcommand{\ec}{\end{center}}
\newcommand{\bfi}{\begin{figure}}
\newcommand{\efi}{\end{figure}}
\newcommand{\ca}[2]{\caption{#1 \label{#2}}}
\newcommand{\ig}[2]{\includegraphics[#1]{#2}}
\newcommand{\spl}[2]{\left\{\begin{array}{ll}#1\\#2\end{array}\right.}
\newcommand{\ie}{{\it i.e.\ }}
\newcommand{\eg}{{\it e.g.\ }}
\newcommand{\pd}[2]{\frac{\partial #1}{\partial #2}}
\newcommand{\pdc}[3]{\left. \frac{\partial #1}{\partial #2}\right|_{#3}}
\newcommand{\infint}{\int_{-\infty}^{\infty} \!\!}      % infinite integral
\newcommand{\tbox}[1]{{\mbox{\tiny #1}}}
\newcommand{\mbf}[1]{{\bm #1}}           % requires bm package
\newcommand{\pO}{{\partial\Omega}}
\newcommand{\uN}{u^{(N)}}
\newcommand{\emach}{\epsilon_\tbox{mach}}
%\newcommand{\crk}{C_{s}} %C_{R,k}         % const for exponential s(m) bnd
\newcommand{\bmp}[1]{\begin{minipage}{#1}}
\newcommand{\emp}{\end{minipage}}
\newcommand{\re}{\mbox{Re}\,}
\newcommand{\im}{\mbox{Im}\,}

\newtheorem{thm}{Theorem}
\newtheorem{cnj}{Conjecture}
\newtheorem{lem}{Lemma}[thm]
\newtheorem{cor}{Corollary}[thm]

\newtheorem{rmk}[thm]{Remark}
\newtheorem{pro}[thm]{Proposition}
%\newtheorem{cnj}[thm]{Conjecture}
%\newtheorem{lemma}[thm]{Lemma}

\newcommand{\bp}{{\bf Proof:\ }}   % {\begin{proof}}
\newcommand{\ep}{\hfill $\square$ \vspace{1ex} \\} % {\end{proof}}
% note elsart ones horrible: \pr and \endpr
\newcommand{\bmal}{{\bm \alpha}}             % bold alpha
\newcommand{\splt}[2]{$\left\{\begin{array}{l}\mbox{#1}\\\mbox{#2}\end{array}\ri
ght.$}
\newcommand{\lo}{{L^2(\Omega)}}
\newcommand{\lpo}{{L^2(\pO)}}
\newcommand{\cu}{\tilde{u}}            % symbol for continuation of u
\newcommand{\cOmega}{\tilde{\Omega}}   % symbol for domain to be continued into
\newcommand{\dmax}{D_\tbox{max}}       % max dist in splane_charge_curve

% spacers...
\newcommand{\tw}{\textwidth}
\newcommand{\vg}{\vspace{2ex}}

% brand names... the problem with putting a trailing space here is that then
% punctuation following the word cannot be correct. To use these with correct
% trailing space, use as \mpspack\ or \matlab\ or with ending period, \matlab.
\newcommand{\mpspack}{{\tt MPSpack}}            % name of this package
\newcommand{\matlab}{{M{\small ATLAB}}}            % name of MATLAB

% code formatting... (nearly ok, gets multiple-line padding wrong)
\newcommand{\co}[1]{\vspace{.5ex}
\\
\mbox{}\hspace{5ex}\begin{minipage}{\textwidth}{\tt #1}\end{minipage}%
\vspace{.5ex}%
\\}



\begin{document}
\title{\mpspack\ tutorial}
\author{Alex Barnett\footnote{Department of Mathematics, Dartmouth College,
Hanover, NH, 03755, USA}
\ and
Timo Betcke\footnote{Department of Mathematics,
University of Reading, Berkshire, RG6 6AX, UK}}
\date{\today}   % how pipe from getrevisionnumber + 1?

\maketitle
\begin{abstract}
This is a short tutorial showing how boundary-value problems
may be numerically solved simply and accurately with the \mpspack\ toolbox
for \matlab. We assume basic familiarity with \matlab\
and with partial differential equations.
%concept of particular-solution type numerical methods is assumed.
\end{abstract}

%\tableofcontents

\section{About this tutorial}

This tutorial is designed for `bottom-up' learning of the features
of \mpspack, i.e.\ by progressing through simple examples.
In that sense it complements the user manual which describes
the theoretical framework in broad strokes and therefore could
be considered `top-down'.
We will skip the mathematics behind the solution techniques,
focusing on computing and plotting useful PDE solutions.

Throughout we will identify the plane $\mathbb{R}^2$ with the complex
plane $\mathbb{C}$, by the usual map $z=x+iy$. In other words
$(2,3)$ and $2+3i$ represent the same point.
We use {\tt teletype} font to designate commands that may be typed
at the \matlab\ prompt.
All the code examples in this document, and code to generate the
figures, is found in {\tt tutorial.m} in the {\tt examples/} directory.

\bfi % ffffffffffffffffffffffffffffffffffffffffffffffffffffffffffffffffffffff
a)\raisebox{-0.3\textwidth}{\ig{height=0.35\textwidth}{seg.eps}}
b)\raisebox{-0.3\textwidth}{\ig{height=0.35\textwidth}{dom.eps}}
\ca{a) circular closed segment, b) unit disc domain.
Both have a periodic trapezoidal quadrature rule
with $M=20$ quadrature points}{f:sd}
\efi

% ----------------------------------------------------------------------------
\section{Solving Laplace's equation in a disc}
\label{s:lap}

We start by setting up a domain in $\mathbb{R}^2$.
% in which a PDE will be solved.
Domains are built from segments which define their boundary.
To make the unit disc domain,
we first need a circle segment with center
0, radius 1, and angle range
$[0,2\pi)$, as follows,
\co{s = segment([], [0 1 0 2*pi])}
The object {\tt s} is indeed a circular segment, as we may check by
typing {\tt s.plot}, producing Fig.~\ref{f:sd}a.
%or by examining its contents by typing {\tt s}.
All segments have a {\em sense}, i.e.\ direction of travel:
for this segment it is counter-clockwise, as shown by the
downwards-pointing
arrow symbol overlayed onto the segment at about 9 o'clock.%
  \footnote{In fact, segments are parametrized internally as function $z(t)$
    of a real variable $t\in[0,1]$, and the sense is the direction of
    increasing $t$. Segment {\tt s} stores this function as {\tt s.Z}.}
Notice also normal vectors (short `hairs') pointing outwards
at each boundary point; our definition is that
normals on a segment always point to the {\em right} when traversing the
sense of the segment.

We create the domain interior to this segment with
\co{d = domain(s, +1)}
where the second argument (here $+1$, the only other option being $-1$)
specifies that the domain is to the `standard' side of the segment, which
we take to be such that the normals point {\em away from} the domain.
That is, with $+1$ the domain lies to the {\em left} of the segment
when traversed in its correct sense (with $-1$ the domain
would lie to the right of the segment.)
Typing {\tt d.plot} produces%
  \footnote{There are extra plotting options and features that
    are described in documentation such as {\tt help domain.plot}.
    E.g.\ in this figure a grid of points interior to the domain has been
    included, achieved with {\tt opts.gridinside=0.05; d.plot(opts);}
    %TODO: demo switch off corner fan.
  }
Fig.~\ref{f:sd}b.
Note that perimeter and area are automatically
labelled (these are only rough approximations intended for sanity checks).

\bfi % ffffffffffffffffffffffffffffffffffffffffffffffffffffffffffffffffffffff
a)\raisebox{-0.3\textwidth}{\ig{height=0.35\textwidth}{u.eps}}
b)\raisebox{-0.3\textwidth}{\ig{height=0.35\textwidth}{uerr.eps}}
\ca{a) Numerical solution field $u$, b) pointwise error $u-f$,
for Laplace's equation in the unit disc with $M=20$ quadrature points
and 8th-order harmonic polynomials.}{f:u}
\efi

Laplace's equation $\Delta u = 0$ is Helmholtz's equation with wavenumber
zero, which we set for this domain with,
\co{d.k = 0;}
Our philosophy is
to approximate the solution in the domain by a linear combination of
{\em basis functions}, each defined over the whole domain.
%We have to choose how many to use, i.e.\ the
%{\em order} of approximation.
We choose 8th-order harmonic polynomials
$u(z) = \sum_{n=0}^{8} c_n \,\mbox{Re}\,z^n +
\sum_{n=1}^{8} c_{-n}\,\mbox{Im}\,z^n$, where $\mbf{c}:=\{c_n\}_{n=-8}^{8}
\in\mathbb{R}^{17}$
is a coefficient vector, based at the origin 0,
%i.e.\ the sets $\{\mbox{Re } z^n\}_{n=0}^{10}$ and
%$\{\mbox{Im } z^n\}_{n=1}^{10}$,
using the command
\co{d.addregfbbasis(0, 8);}
Let's specify Dirichlet boundary data $f(z) = 
\ln |z-2-3i|$ for
$z$ on the segment%
  \footnote{In other words, $f(x,y) = \ln \sqrt{(x-2)^2+(y-3)^2}$
    for points $(x,y)$ on the boundary.}
by representing this as an anonymous function {\tt f}
and associating it with one side of the segment,
\begin{verbatim}
   f = @(z) log(abs(z-2-3i));
   s.setbc(-1, 'd', [], @(t) f(s.Z(t)));
\end{verbatim}
Note that we needed to pass in a function not of location $z$,
but of the segment parameter $t$;
this was achieved by
wrapping ${\tt f}$ around the parametrization function ${\tt s.Z}$.
The first argument $-1$ expresses that the boundary condition is to be
understood in the limit approaching from the side {\em opposite} the
segment's normal direction, which is where the domain is located.
The second argument {\tt 'd'} specifies that the data is Dirichlet.
 
Finally we use the domain to make a boundary-value
problem object {\tt p},
\co{p = bvp(d);}
and may then solve (in the least-squares sense)
a linear system for the coefficients
\co{p.solvecoeffs;}
If it is needed, {\tt p.co} now contains the coefficients vector $\mbf{c}$.
To evaluate and plot the solution we simply use,
\co{p.showsolution;}
The software chose an appropriate grid covering the domain
(points outside the domain are made transparent), giving Fig.~\ref{f:u}a.

\bfi % ffffffffffffffffffffffffffffffffffffffffffffffffffffffffffffffffffffff
a)\raisebox{-0.3\textwidth}{\ig{height=0.35\textwidth}{N.eps}}
b)\raisebox{-0.3\textwidth}{\ig{height=0.35\textwidth}{radfunc.eps}}
\ca{a) Convergence of boundary error $L^2$ norm for harmonic
polynomials for Laplace equation
in the unit disc, b) solution error for same boundary data $f$
in a smooth star-shaped domain (normals also shown).}{f:conv}
\efi

% ----------------------------------------------------------------------------
\section{Accuracy, convergence, and smooth domains}

How accurate was our numerical solution $u$? One measure is the
$L^2$ error on the boundary, and is estimated by
\co{p.bcresidualnorm}
which returns $2.09 \times 10^{-6}$.
However, since the function $f(z)$ is already harmonic in the domain,
it is in fact the unique solution, and we may plot the
pointwise error in $u$ by passing in the analytic solution as an option,
\co{opts.comparefunc = f; p.showsolution(opts);}
giving Fig.~\ref{f:u}b. Note that the color scale is $10^{-8}$.

In the above, boundary integrals were approximated using the default of
$M=20$ quadrature points, barely adequate given the
oscillatory error function in Fig.~\ref{f:u}b.
$M$ may be easily changed either by specifying
a non-empty first argument in the {\tt segment} constructor above, or
for an existing segment as follows,
\co{s.requadrature(50); p.solvecoeffs; p.bcresidualnorm}
which now gives $1.98\times 10^{-6}$, not much different than before.
Notice that we did not have to redefine the domain {\tt d} nor
the BVP object {\tt p}.

Exploring the convergence of the boundary error norm with the basis set order
needs a simple loop and figure,
\begin{verbatim}
   for N=1:15
     d.bas{1}.N = N; p.solvecoeffs; r(N) = p.bcresidualnorm;
   end
   figure; semilogy(r, '+-'); xlabel('N'); ylabel('bdry err norm');
\end{verbatim}
As Fig.~\ref{f:conv}a shows, the convergence is exponential.%
  \footnote{Asymptotically, error $\sim e^{-\alpha N}$. In fact the rate is
    $\alpha = \ln \sqrt{13}$, related to the conformal distance to
    the nearest singularity \cite{timothesis}, which here is at $2+3i$.}

Say we want to change the shape of segment {\tt s}, to
a smooth star-shaped domain expressed as by radius $R(\theta) =
1 + 0.3\cos 3\theta$ as a function of angle $0\le \theta< 2\pi$.
This is achieved by passing a 1-by-2
cell array containing the function $R$ and its
derivative $R' = dR/d\theta$ to a variant of the segment constructor,
\begin{verbatim}
   s = segment.radialfunc(50, {@(q) 1 + 0.3*cos(3*q), @(q) -3*0.3*sin(3*q)});
\end{verbatim}
We again chose $M=50$.
The analytic formula for $R'$ is needed to compute normal derivatives
to high accuracy.

One might ask: has this change to {\tt s} {\em propagated}
to the existing domain
object {\tt d} and BVP object {\tt p}, which both refer to it?
In contrast to the case of quadrature point number $M$ above,
the answer is no:
{\tt s} is overwritten by a newly-constructed object, while
{\tt d} and {\tt p} still contain handles pointing to the {\em old}
{\tt s}.
Furthermore, the fact that the segment had domain {\tt d}
attached to its `minus' or back side has been forgotten, as have the
boundary conditions.
(These segment properties are described in the \mpspack\ user manual.)
We must therefore rerun the code from Sec.~\ref{s:lap}
to construct {\tt d} and {\tt p} afresh, before solving.%
  \footnote{Note that in theory it would be possible to
    change one by one each of the segment properties, {\tt t}, {\tt w},
    {\tt speed}, etc, to define the new segment without changing its identity,
    but this is cumbersome. Similarly, searching and changing
    all references to a segment in the properties of {\tt d} and {\tt p}
    is cumbersome. Neither has been implemented since problem setup time is
    very rapid.}
The result, plotting the pointwise error as before,
is shown by Fig.~\ref{f:conv}b for $N=8$ and $M=50$.

The {\tt radialfunc} constructor above is limited to radial functions
with quadrature equidistant
in angle. Instead you may create a segment from arbitrary
smooth parametrizations $z(t)$ for $t \in[0,1]$, as long as $z'(t)$
is also given. For instance, a closed crescent-shaped analytic segment is
produced by 
\begin{verbatim}
   a = 0.2; b = 0.8; w = @(t) exp(2i*pi*t);
   s = segment(100, {@(t) w(t)-a./(w(t)+b), ...
                     @(t) 2i*pi*w(t).*(1 + a./(w(t) + b).^2)}, 'p');
\end{verbatim}
Note the nested anonymous functions for mathematical clarity.
Note also the new final argument {\tt 'p'} which enforces
periodic quadrature (the constructor doesn't try to guess your preferred rule).
In order to get high-order (or spectral) convergence, it is recommended
that you choose only smooth (or analytic) $z$.
If periodic quadrature is used,
this also applies to the 1-periodic extension of $z$ to the real line.
If $z(1)\neq z(0)$, the ends of the segment will not connect
up, and the domain constructor above will report an error.


% ----------------------------------------------------------------------------
\section{Helmholtz equation, exterior and non-simply connected domains}

Changing from the Laplace to Helmholtz equation is as simple
as setting {\tt d.k} to a different value.
We start a fresh example: a exterior Helmholtz BVP
with Neumann boundary data, and the Sommerfeld radiation
condition \cite{coltonkress}. This has a unique solution.





\begin{verbatim}
   
\end{verbatim}

\begin{verbatim}
   
\end{verbatim}






% ----------------------------------------------------------------------------
\section{Polygons and domains with corners}

line segments, quadrature rules on $[0,1]$.

corners in domains.

The small black semicircle at angle $\theta=0$ illustrates that \mpspack\
has connected the start and end of the segment to form a `corner'.
Although we did not discuss it, this also happens when a segment
is connected to itself as in Sec.~\ref{s:lap}.



% ----------------------------------------------------------------------------
\section{Scattering and transmission problems}

% ----------------------------------------------------------------------------
\section{Corner basis sets}

% ----------------------------------------------------------------------------
\section{Layer potentials}




\bibliographystyle{siam} 
\bibliography{alexrefs}
\end{document}
